%! Author = robmandelings
%! Date = 02/12/2022

% Preamble
\documentclass[11pt]{article}

% Packages
\usepackage{amsmath}
\usepackage[left=20mm, right=20mm]{geometry}
\bibliographystyle{plain}

% Document
\begin{document}

    \section{Introduction}


    \section{Board Encoding}

    In order to create an efficient information retrieval system over chess games, we need to develop appropriate indexing for the documents. Therefore, we have to find a way to encode board configuration so that the important information is preserved. We have used several encoding techniques, each of which highlights different features of a specific board configuration.

    \subsection{Board State}

    The first encoding is the most straightforward, where each piece on the board is encoded. Every piece encoding consists of a piece type, color and position on the board. An example can be seen on (placeholder)

    \subsubsection{Reachability}

    This encoding converts the board state into a list of reachable squares, for every piece on the board. A reachable square for a piece is considered to be a square where this piece is allowed to move to on its turn. Board positions that have reachabilities similar to the query board position is regarded as a better match.

    Another idea included in the reachability is the distance between the piece and the reachable square. The larger the distance, the smaller the weight that is given to this reachable square (see ~\eqref{eq:distanceMetric} for the specific metric that was used). An example is given on figure (figure).
    % TODO reachability only necessary to be encoded in the board

    \begin{equation}
        \label{eq:distanceMetric}
    \end{equation}

    \subsubsection{Attack}

    Another encoding is attack, where the board is converted into a sequence of squares with opponents that can be attack by another piece. This encoding is performed for each piece on the board, and concatenated as a single string. You can see an example of this on figure ()

    \subsubsection{Defense}

    This is similar to the attack encoding, except that the squares should now contain pieces of the same color. Thus, each piece that \'attacks\' a piece of its own color, actually defends it, as the defended pieces cannot be taken without risk. An example is given at figure ().

    \subsubsection{Ray-Attack}

    \subsubsection{Pinning}

    \subsubsection{Other possibilities}

    We have only implemented a sample of the possible encodings for these board configurations, as there is other useful information that might be encoded from these boards to get even more relevancy.

    % TODO er zou in principe ook met de parameters kunnen worden gespeeld om sommige encoderingen een groter gewicht / belang te laten hebben dan anderen om de relevancy te verhogen (bijvoorbeeld check of checkmate).


    \section{Document format}

    % TODO full document format example include here


    \section{Solr}

    \subsection{BK25}

    \sections{Evaluation}

    \sections{Indexing Speed}

    % TODO if there are boards that are exactly the same as the query, these should be retrieved and visible at the top (one evaluation)

    % TODO can be hard to evaluate the relevance, if we would elaborate on this project even more, we would have users judge the relevance of chess board retrievals 

    \bibliography{refs}

\end{document}